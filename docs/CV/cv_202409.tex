% Options for packages loaded elsewhere
\PassOptionsToPackage{unicode}{hyperref}
\PassOptionsToPackage{hyphens}{url}
\PassOptionsToPackage{space}{xeCJK}
%
\documentclass[
  12pt,
]{article}
\usepackage{amsmath,amssymb}
\usepackage{lmodern}
\usepackage{iftex}
\ifPDFTeX
  \usepackage[T1]{fontenc}
  \usepackage[utf8]{inputenc}
  \usepackage{textcomp} % provide euro and other symbols
\else % if luatex or xetex
  \usepackage{unicode-math}
  \defaultfontfeatures{Scale=MatchLowercase}
  \defaultfontfeatures[\rmfamily]{Ligatures=TeX,Scale=1}
  \setmainfont[]{Times New Roman}
  \ifXeTeX
    \usepackage{xeCJK}
    \setCJKmainfont[]{MS Gothic}
  \fi
  \ifLuaTeX
    \usepackage[]{luatexja-fontspec}
    \setmainjfont[]{MS Gothic}
  \fi
\fi
% Use upquote if available, for straight quotes in verbatim environments
\IfFileExists{upquote.sty}{\usepackage{upquote}}{}
\IfFileExists{microtype.sty}{% use microtype if available
  \usepackage[]{microtype}
  \UseMicrotypeSet[protrusion]{basicmath} % disable protrusion for tt fonts
}{}
\makeatletter
\@ifundefined{KOMAClassName}{% if non-KOMA class
  \IfFileExists{parskip.sty}{%
    \usepackage{parskip}
  }{% else
    \setlength{\parindent}{0pt}
    \setlength{\parskip}{6pt plus 2pt minus 1pt}}
}{% if KOMA class
  \KOMAoptions{parskip=half}}
\makeatother
\usepackage{xcolor}
\usepackage[margin=1in]{geometry}
\usepackage{graphicx}
\makeatletter
\def\maxwidth{\ifdim\Gin@nat@width>\linewidth\linewidth\else\Gin@nat@width\fi}
\def\maxheight{\ifdim\Gin@nat@height>\textheight\textheight\else\Gin@nat@height\fi}
\makeatother
% Scale images if necessary, so that they will not overflow the page
% margins by default, and it is still possible to overwrite the defaults
% using explicit options in \includegraphics[width, height, ...]{}
\setkeys{Gin}{width=\maxwidth,height=\maxheight,keepaspectratio}
% Set default figure placement to htbp
\makeatletter
\def\fps@figure{htbp}
\makeatother
\setlength{\emergencystretch}{3em} % prevent overfull lines
\providecommand{\tightlist}{%
  \setlength{\itemsep}{0pt}\setlength{\parskip}{0pt}}
\setcounter{secnumdepth}{-\maxdimen} % remove section numbering
\usepackage{xeCJK}
\usepackage{enumitem}
\ifLuaTeX
  \usepackage{selnolig}  % disable illegal ligatures
\fi
\IfFileExists{bookmark.sty}{\usepackage{bookmark}}{\usepackage{hyperref}}
\IfFileExists{xurl.sty}{\usepackage{xurl}}{} % add URL line breaks if available
\urlstyle{same} % disable monospaced font for URLs
\hypersetup{
  pdftitle={Curriculum Vitae},
  pdfauthor={Keisuke Hanada},
  hidelinks,
  pdfcreator={LaTeX via pandoc}}

\title{Curriculum Vitae}
\author{Keisuke Hanada}
\date{2024-09-06}

\begin{document}
\maketitle

\hypertarget{contact-information}{%
\subsection{\texorpdfstring{\textbf{Contact
Information}}{Contact Information}}\label{contact-information}}

\begin{itemize}
\tightlist
\item
  Name: Keisuke Hanada
\item
  Address: Graduate School of Engineering Science, Osaka University
\item
  Email: hanada.keisuke.es {[}at{]} osaka-u.ac.jp
\item
  ORCID: 0000-0002-1444-0280
\item
  GitHub: \url{https://github.com/keisuke-hanada}
\item
  Website: \url{https://keisuke-hanada.github.io/}
\end{itemize}

\begin{center}\rule{0.5\linewidth}{0.5pt}\end{center}

\hypertarget{working-experience}{%
\subsection{Working Experience}\label{working-experience}}

\begin{itemize}[left=80pt, labelsep=1em, itemindent=0pt, label={}]
  \item[2024/4 - ~~~~~~~~~~~] Specially Appointed Assistant Professor,  \\
  Graduate School of Engineering Science, Osaka University
  \item[2019/4 - 2024/2] Biostatistician, \\
  Biometrics Department, Kyowa Kirin Co., Ltd
\end{itemize}

\begin{center}\rule{0.5\linewidth}{0.5pt}\end{center}

\hypertarget{education}{%
\subsection{Education}\label{education}}

\begin{itemize}[left=80pt, labelsep=1em, itemindent=0pt, label={}]
  \item[2020/4 - 2024/3] Ph.D. in Data Science, Shiga University \\
  Supervisor: Dr. Tomoyuki Sugimoto
  \item[2017/4 - 2019/3] M.S. in 数理情報工学科, Kagoshima University \\
  Supervisor: Dr. Tomoyuki Sugimoto
  \item[2013/4 - 2017/3] B.S. in 数理科学科, Hirosaki University  \\
  Supervisor: Dr. Kimitoshi Tsutaya
\end{itemize}

\begin{center}\rule{0.5\linewidth}{0.5pt}\end{center}

\hypertarget{research-publications}{%
\subsection{Research Publications}\label{research-publications}}

\hypertarget{peer-reviewed-articles}{%
\subsubsection{Peer-Reviewed Articles}\label{peer-reviewed-articles}}

\begin{enumerate}
\def\labelenumi{\arabic{enumi}.}
\tightlist
\item
  \underline{Hanada, K}., Moriya, J., \& Kojima, M. (2024). Comparison
  of baseline covariate adjustment methods for restricted mean survival
  time. \emph{Contemporary Clinical Trials}, 138, 107440.
\item
  \underline{Hanada, K}., \& Sugimoto, T. (2023). Inference using an
  exact distribution of test statistic for random-effects meta-analysis.
  \emph{Annals of the Institute of Statistical Mathematics}, 75(2),
  281-302.
\end{enumerate}

\hypertarget{preprints}{%
\subsubsection{Preprints}\label{preprints}}

\begin{enumerate}
\def\labelenumi{\arabic{enumi}.}
\tightlist
\item
  \underline{Hanada, K}., \& Sugimoto, T. (2024). Random-Effect
  Meta-Analysis with Robust Between-Study Variance. \emph{arXiv
  preprint} arXiv:2407.04446.
\item
  \underline{Hanada, K}., \& Kojima, M. (2024). Bayesian Parametric
  Methods for Deriving Distribution of Restricted Mean Survival Time.
  \emph{arXiv preprint} arXiv:2406.06071.
\item
  \underline{Hanada, K}., \& Kojima, M. (2024). Random Effect Restricted
  Mean Survival Time Model. \emph{arXiv preprint} arXiv:2401.02048.
\item
  Kojima, M., Mano, H., Yamada, K., \underline{Hanada, K}., Tanaka, Y.,
  \& Moriya, J. (2023). Adjusting confidence intervals under
  covariate-adaptive randomization in non-inferiority and equivalence
  trials. \emph{arXiv preprint} arXiv:2312.15619.
\end{enumerate}

\hypertarget{software-and-r-packages}{%
\subsubsection{Software and R Packages}\label{software-and-r-packages}}

\begin{enumerate}
\def\labelenumi{\arabic{enumi}.}
\tightlist
\item
  rmstBayespara: Bayesian Restricted Mean Survival Time for Cluster
  Effect. R package. Available at:
  \url{https://cran.r-project.org/web/packages/rmstBayespara/}
\item
  metaMest: meta-analysis by M-estimator based approach. R package.
  Available at: \url{https://github.com/keisuke-hanada/metaMest}
\end{enumerate}

\hypertarget{conference-presentations}{%
\subsubsection{Conference
Presentations}\label{conference-presentations}}

\begin{enumerate}
\def\labelenumi{\arabic{enumi}.}
\tightlist
\item
  \underline{Hanada, K}., \& Sugimoto, T. (2019). Non-Asymptotic
  Properties and Behaviors for Random-Effects Meta-Analysis When the
  Number of Studies Is Small'', \emph{The VI-th International Symposium
  on Biopharmaceutical Statistics}, Kyoto.
\end{enumerate}

\begin{center}\rule{0.5\linewidth}{0.5pt}\end{center}

\hypertarget{grants-and-funding}{%
\subsection{Grants and Funding}\label{grants-and-funding}}

\begin{itemize}[left=80pt, labelsep=1em, itemindent=0pt, label={}]
  \item[2024/7 - 2026/3] Japan Society for the Promotion of Science (JSPS),     
Grant-in-Aid for Research Activity Start-up [Principal Investigator]
  \item[2024/4 - 2027/3] Japan Society for the Promotion of Science (JSPS),     
Grant-in-Aid for Scientific Research (C) [Co-Investigator]
\end{itemize}

\begin{center}\rule{0.5\linewidth}{0.5pt}\end{center}

\hypertarget{academic-service}{%
\subsection{Academic Service}\label{academic-service}}

\begin{itemize}
\tightlist
\item
  \textbf{Peer Review}\\
  Japanese Kournal of Statistics and Data Science, 2024\\
  Journal of the Royal Statistical Society: Series C (Applied
  Statistics), 2024
\end{itemize}

\begin{center}\rule{0.5\linewidth}{0.5pt}\end{center}

\hypertarget{research-publications-in-japanese}{%
\subsection{Research Publications (in
Japanese)}\label{research-publications-in-japanese}}

\hypertarget{ux539fux8457ux8ad6ux6587}{%
\subsubsection{原著論文}\label{ux539fux8457ux8ad6ux6587}}

\begin{enumerate}
\def\labelenumi{\arabic{enumi}.}
\tightlist
\item
  \underline{花田圭佑}, \& 杉本知之. (2024).
  イベント時間アウトカムに対する個人データ復元と要約統計量に基づくメタアナリシスとその性能.
  \emph{計量生物学}, 45(1), 115-131.
\end{enumerate}

\hypertarget{ux5b66ux4f1aux767aux8868}{%
\subsubsection{学会発表}\label{ux5b66ux4f1aux767aux8868}}

\begin{enumerate}
\def\labelenumi{\arabic{enumi}.}
\tightlist
\item
  \underline{花田圭佑}, \& 小島将裕. (2024). Random Effect Restricted
  Mean Survival Time Model, 2024年度統計関連学会連合大会.
\item
  \underline{花田圭佑}, \& 杉本知之. (2024).
  少数試験メタアナリシスでの分散推定, 日本計算機統計学会第38回大会.
\item
  \underline{花田圭佑}. (2023).
  イベント時間アウトカムの個人データ復元とメタアナリシス,
  第5回かごしまデータ科学シンポジウム.
\item
  \underline{花田圭佑}, 小島将裕, \& 守屋順之. (2023).
  制限付き平均生存時間 (RMST) の共変量調整法の比較,
  2023年度統計関連学会連合大会.
\item
  \underline{花田圭佑}. (2022).
  イベント数が少ないもとでの二値メタアナリシスの推測,
  第3回かごしまデータ科学シンポジウム.
\item
  \underline{花田圭佑}, \& 杉本知之. (2019).
  試験数が少ない場合のランダム効果メタアナリシスの非漸近的性質,
  日本計算機統計学会第33回シンポジウム.
\item
  \underline{花田圭佑}, \& 杉本知之. (2018).
  ランダム効果DL層別解析における標本分布,
  日本計算機統計学会第32回シンポジウム.
\item
  \underline{花田圭佑}, \& 杉本知之. (2018).
  正規母集団におけるランダム効果層別解析における標本分布,
  大分統計談話会第57回大会.
\end{enumerate}

\begin{center}\rule{0.5\linewidth}{0.5pt}\end{center}

\end{document}
